\documentclass[12pt,a4paper]{article}
\usepackage[utf8]{inputenc}
\usepackage{amsmath}
\DeclareMathOperator\arctanh{arctanh}
\usepackage{setspace}
\renewcommand{\baselinestretch}{2} 
\usepackage[left=2cm,right=2cm,top=2cm,bottom=2cm]{geometry}
\begin{document}
\begin{center}
\begin{Huge}
3D Lorentz transzformáció osztálya\\
\end{Huge}
\begin{large}
Sróka András\\
\end{large}
\end{center}
A Lorentz transzformációk olyan transzformációk amikre a
\begin{equation}
I = x_0^2-x_1^2-x_2^2-x_3^2
\end{equation}
ívhossz értéke invariáns.\\
Az egyik alapvető ilyen művelet az x tengely menti Lorentz boost, amit a következő egyenletek írnak le:
\renewcommand{\baselinestretch}{0.5}
\begin{equation}
t' = \frac{\gamma \left( ct-\beta x \right)}{c}$$$$
x' = \gamma \left( x-\beta ct \right)$$$$
y'=y$$$$
z'=z
\end{equation}
\renewcommand{\baselinestretch}{2} 
ahol $\gamma = \frac{1}{\sqrt{1-\beta^2}}$, $\beta  = \frac{v}{c}$, $c$ a fény sebessége és $v$ az eredetihez lépest mozgó koordináta rendszer sebessége.\\
Ezt lehet általánosítani egy tetszőleges irányú transzformációvá egy mátrixszorzással:
$$X'=L(v)X$$
ahol\begin{center}
$X = \begin{pmatrix}
ct\\x\\y\\z
\end{pmatrix}
$   és   $
X' = \begin{pmatrix}
ct'\\x'\\y'\\z'
\end{pmatrix}$
\end{center}
Ekkor a transzformáció $L(v)$ mátrixa
\begin{equation}
L(v) = \begin{pmatrix}
\gamma & -\gamma\frac{v_x}{c} & -\gamma\frac{v_y}{c} & -\gamma\frac{v_z}{c}\\
-\gamma\frac{v_x}{c} & 1+(\gamma-1)\frac{v_x^2}{v^2} & (\gamma-1)\frac{v_xv_y}{v^2} & (\gamma-1)\frac{v_xv_z}{v^2}\\
-\gamma\frac{v_y}{c} & (\gamma-1)\frac{v_xv_y}{v^2} & 1+(\gamma-1)\frac{v_y^2}{v^2} & (\gamma-1)\frac{v_yv_z}{v^2}\\
-\gamma\frac{v_z}{c} & (\gamma-1)\frac{v_xv_z}{v^2} & (\gamma-1)\frac{v_yv_z}{v^2} & 1+(\gamma-1)\frac{v_z^2}{v^2}\\
\end{pmatrix}
\end{equation}
ahol $v_i$ a sebesség i irányú komponense, $v$ pedig a sebességvektor nagysága.\\
Lorentz transformációk a forgatások is, melyeket a
\begin{equation}
R_x(\theta)=\begin{pmatrix}
1 & 0 & 0 & 0 \\
0 & 1 & 0 & 0 \\
0 & 0 & \cos(\theta) & \sin(\theta) \\
0 & 0 & -\sin(\theta) & \cos(\theta) \\
\end{pmatrix}
R_y(\theta)=\begin{pmatrix}
1 & 0 & 0 & 0 \\
0 & \cos(\theta) & 0 & \sin(\theta) \\
0 & 0 & 1 & 0 \\
0 & -\sin(\theta) & 0 & \cos(\theta) \\
\end{pmatrix}$$$$
R_z(\theta)=\begin{pmatrix}
1 & 0 & 0 & 0 \\
0 & \cos(\theta) & \sin(\theta) & 0 \\
0 & -\sin(\theta) & \cos(\theta) & 0 \\
0 & 0 & 0 & 1 \\
\end{pmatrix}
\end{equation}
mátrixok írnak le, ahol $\theta$ az adott tengley körüli forgatás szöge.\\
\bigskip
Emellett Lorentz transzformációk a tér-, és időtükrözés transzformációk is.\\
A Lorentz boostot le lehet írni egy hiperbolikus forgatásként is, aminek a szöge $\alpha=\arctanh \left(\frac{v}{c}\right)$, ahol $\alpha$ a rapiditás.
\end{document}